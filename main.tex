% This is based on the LLNCS.DEM the demonstration file of
% the LaTeX macro package from Springer-Verlag
% for Lecture Notes in Computer Science,
% version 2.4 for LaTeX2e as of 16. April 2010
%
% See http://www.springer.com/computer/lncs/lncs+authors?SGWID=0-40209-0-0-0
% for the full guidelines.
%
\documentclass{llncs}

% make a proper TOC despite llncs
\setcounter{tocdepth}{2}
\makeatletter
\renewcommand*\l@author[2]{}
\renewcommand*\l@title[2]{}
\makeatletter

\usepackage[utf8]{inputenc}
\usepackage[nolist,nohyperlinks]{acronym}
\usepackage[stable]{footmisc}
\usepackage{mathtools}
\usepackage{rotating}
\usepackage{pifont}
\usepackage{placeins}
\usepackage{fancyhdr}
\usepackage{subfig}

\newcommand{\cmark}{\ding{51}}
\newcommand{\xmark}{\ding{55}}
\setcounter{secnumdepth}{3}
\setcounter{tocdepth}{3}
\setlength{\arrayrulewidth}{0.2pt}%.4pt
{\renewcommand{\arraystretch}{1.2}
\pagestyle{plain}
\setcounter{page}{1}
\pagenumbering{arabic}
\fancyhf{}
\fancyfoot[R]{\thepage}
\captionsetup{belowskip=12pt,aboveskip=4pt}

\begin{document}

\title{Power-Aware Security Protocols\\ for the Internet of Things}
%
\titlerunning{}  % abbreviated title (for running head)
%                                     also used for the TOC unless
%                                     \toctitle is used
%
\author{Tiago Miguel Correia Diogo}
%
\authorrunning{Tiago Diogo} % abbreviated author list (for running head)
%
%%%% list of authors for the TOC (use if author list has to be modified)
\tocauthor{Tiago Diogo}
%
\institute{Instituto Superior T\'ecnico, Universidade de Lisboa\\ Avenida Rovisco Pais 1, Lisboa,\\
\email{tiago.diogo@tecnico.ulisboa.pt}}

\maketitle              % typeset the title of the contribution

\begin{abstract}
The \ac{IoT} and its vision of connecting every device to one another presents an opportunity to create large information sharing networks. However, intruders can take advantage of the \ac{IoT} devices constrained nature to disrupt the networks and launch a wide range of attacks on its nodes. In our work we address this issue from a power-aware perspective, trying to find the best relation between security and power consumption. To achieve this objective we do a thoroughly analysis of the existing protocols, attacks and mitigation strategies, combining that information into our proposed network management system to be evaluated on a Smart Campus scenario. Furthermore, we will perform energy consumption profiling to endow future users with the knowledge of what kind of physical resources to deploy, based on the desired application security level.
\keywords{Internet of Things, Power-Aware Security, Secure Bootstrapping, CoAP, MQTT, 6LoWPAN, RPL, IEEE 802.15.4}
\end{abstract}

\tableofcontents
\newpage

\section{Introduction}
\paragraph{
The \ac{IoT} can be seen as web of interconnected devices that go from everyday wearable objects into fully deployed sensor networks. Despite the huge variety and characteristics of these devices, one thing that they all have in common in the constrained nature they're built uppon. In order to enable the massive deploy to be expected in the near future \footnote{http://www.gartner.com/newsroom/id/2636073} \ac{IoT} devices must be accessible and affordable, capable of operating under lossy wireless networks while being battery powered.
}


\section{Related Work}
\label{sec:related_work}
\subsection{Protocol Analysis and Selection}
some text here			
\subsubsection{Web Protocols}
some more text here
\subsubsection{IoT Protocols}
and some more text here
\subsubsection{IoT Protocols Security and Improvements}
\subsection{Attack Analysis, Detection and Prevention}
\subsubsection{Internet Attacks}
\cite{Bose2015}
\subsubsection{IoT Attacks}

\section{Proposed Solution}
\label{sec:proposed_solution}
% Quais os elementos principais da solucao proposta.


vou falar de:
 ter esta stack bonitinha que os outros ainda não têm
 preocupar em resolver este ataques ao nível do routing por meter lá as chaves logo
 falar que o custo de meter lá as chaves conpensa na segurança do sistema depois
 clone ficam resolvidos garantindo que um atacante não consegue extrair a informação dos nós.
 


\subsection{Architecture}
\subsection{Protocol Analysis and Selection}

\begin{figure}[h]
  \centering
  \includegraphics[width=0.8\linewidth]{figures/Global_Architecture.png}
  \caption{Global System Architecture}
  \label{fig:global_architecture}
\end{figure}

\section{Work Evaluation}
\label{sec:work_evaluation}
Following the power-aware perspective of this work, our solution will evaluate the power consumption of the system in several scenarios with different network configurations:

\begin{itemize}
	\item No Security: The system does not provide any type of security credentials. All messages are exchanged in plain text and no node authentication is performed. This will be the baseline.\\
	\item Shared Key: The system provides to new nodes a shared group key that enables them to join a secure instance of the network layer protocol RPL therefore assuring node authentication at the network layer.\\
	\item Asymmetric Cryptography : The system provides to new nodes an asymmetric key pair and the client observer public key. This enables the new nodes to join a secure instance of the application layer protocol \ac{CoAP} therefore assuring node authentication at the application layer. Moreover, this enables the \ac{DTLS} handshake to be performed using raw public keys assuring message confidentiality and integrity.\\
	\item Full Security Credentials: The system provides to new nodes both the shared group key, the asymmetric key pair and the client observer public key. 
\end{itemize}

After the data is collected, it will be analysed and charted so that the added power consumption of inserting security measures can be traced to the increasing power consumption. This will allow a network administrator to consider the type of device and powering mechanisms to deploy based on the security level he desires for a given application.

\section{Work Planning}
\label{sec:work_planning}
The work on the proposed solution is guided by the following schedule:

\begin{itemize}
	\item January 9\textsuperscript{th} 2016 - January 31\textsuperscript{st}: Search for the most suitable protocol and operating system implementations for constrained devices.
	\item February 1\textsuperscript{st} 2016 - February 15\textsuperscript{th}: Implement, Test and Debug the proposed solution on a Network Simulator (first version).
	\item February 16\textsuperscript{st} 2016 - February 29\textsuperscript{th}: Implement, Test and Debug the proposed solution on a Network Simulator (improved version)
	\item March 1\textsuperscript{st} - March 15\textsuperscript{th}: Implement, Test and Debug on a Physically Constructed Network (first version).
	\item March 16\textsuperscript{th} - March 31\textsuperscript{th}: Implement, Test and Debug on a Physically Constructed Network (improved version).
	\item April 1\textsuperscript{st} - April 11\textsuperscript{th}: Measure, profile and document energy consumptions for the previously described test cases.
	\item April 12\textsuperscript{th} - May 12\textsuperscript{th}: MSc Thesis Writing
	\item May 13\textsuperscript{th}: MSc Thesis Delivery
\end{itemize}

\section{Conclusion}
\label{sec:conclusion}
Due to the limitations of \ac{IoT} devices, achieving secure communications is not an easy task. In order to allow the deployment of battery powered nodes, their communication model must be very efficient and consume the minimum amount of power required for operation. To achieve those requirements we started by analysing the existing protocols across the OSI layers, trying to find the best suited solutions for this type of environments. After a thorough comparison we achieved a working stack of protocols but soon discovered possible breaches and attacks, especially on the network layer. Those attacks were further investigated and catalogued. Given the common principle on the majority of the attacks, the introduction of rogue nodes to the network, we presented some possible solutions based on secure bootstrapping, the secure authentication of new nodes when joining a network.
Once the energy efficient stack, possible attacks and mitigation strategies were defined, we proposed our solution based on a Smart Campus scenario. This solution is focused on providing the joining devices all the secure credentials required for a secure bootstrapping before the deploy on the field, so that when they start the operation phase no additional credentials need to be fetched, implying that no additional energy is spent on configuration.
Always maintaining a power-aware perspective, the system will be evaluated by measuring its energy consumption with different configurations. These range from ``no security'' where messages are sent in plain text and no node authentication is performed, to ``full security'', where the node is authenticated at network and application layers and messages are sent cyphered. This charting allows future users of the system to decide the type of resources they need to allocate in order to achieve a desired level of security for their application. As future work, currently out of the scope of this project, memory access protection should be addressed in order to the prevent the stealing of secure credentials from deployed devices.

%
% ---- Bibliography ----

\bibliographystyle{splncs}
\bibliography{references}

% *** DEFINITION OF ACRONYMS ***
\acrodef{IoT}{Internet of Things}
\acrodef{MQTT}{Message Queue Telemetry Transport}
\acrodef{MQTT-SN}{Message Queue Telemetry Transport for Sensor Networks}
\acrodef{CoAP}{Constrained Application Protocol}
\acrodef{HTTP}{Hypertext Transfer Protocol}
\acrodef{TLS}{Transport Layer Security}
\acrodef{DTLS}{Datagram Transport Layer Security}
\acrodef{WWW}{World Wide Web}
\acrodef{MTU}{Maximum Transmission Unit}
\acrodef{TCP}{Transmission Control Protocol}
\acrodef{REST}{REpresentational State Transfer}
\acrodef{UDP}{User Datagram Protocol}
\acrodef{URIs}{Universal Resource Identifiers}
\acrodef{M2M}{Machine to Machine}
\acrodef{QoS}{Quality of Service}
\acrodef{WLAN}{Wireless Local Area Networks}
\acrodef{WPAN}{Wireless Private Area Networks}
\acrodef{LR-WPAN}{Low-Rate Wireless Private Area Networks}
\acrodef{MAC}{Medium Access Control}
\acrodef{FFD}{Full Function Device}
\acrodef{RFD}{Reduced Function Device}
\acrodef{IETF}{Internet Engineering Task Force}
\acrodef{DoS}{Denial of Service}
\acrodef{DODAG}{Destination Oriented Directed Acyclic Graph}
\acrodef{DIO}{DODAG Information Objects}
\acrodef{DAO}{Destination Advertisement Objects}
\acrodef{DIS}{DODAG Information Solicitation}
\acrodef{RFID}{Radio Frequency Identification}
\acrodef{ACL}{Access Control List}
\acrodef{CSDS}{\ac{CoAP} Service Discovery Server}
\acrodef{IST}{Instituto Superior Técnico}
\acrodef{6LoWPAN}{IPv6 over Low power Wireless Personal Area Networks}
\acrodef{RPL}{Routing Protocol for Low-Power and Lossy Networks}
\acrodef{CA}{Certificate Authority}
\end{document}
